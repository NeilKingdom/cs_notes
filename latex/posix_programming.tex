\documentclass{article}

\usepackage{titling}
\usepackage{geometry}
\usepackage{fontspec}
\usepackage{color}
\usepackage{xcolor}
\usepackage{tcolorbox}
\usepackage{fancyhdr}
\usepackage{hyperref}
\usepackage{tabularx}
\usepackage[skip=11pt]{parskip}
\usepackage[outputdir=/home/neil/.local/share/latex/output]{minted}

\definecolor{bg}{RGB}{22,43,58}

\hypersetup{
    colorlinks=true,
    urlcolor=blue
}

\tcbuselibrary{listings, minted, skins}
\tcbset{listing engine=minted}

% C codeblocks
\newtcblisting{clst}{%
   listing only,
   minted language=c,
   minted style=monokai,
   colback=bg,
   enhanced,
   frame hidden,
   minted options={%
      tabsize=4,
      breaklines,
      autogobble
   }
}

\hypersetup{%
   colorlinks=true,
   linktoc=all,
   linkcolor=gray,
}

\setmainfont{LiberationSans}
\geometry{%
  margin=1in
}

\fancyhf{}
\renewcommand{\headrulewidth}{0pt}
\fancyfoot[R]{\thepage}
\pagestyle{fancy}
\fancypagestyle{plain}{\pagestyle{fancy}}

\renewcommand\maketitlehooka{\null\mbox{}\vfill}
\renewcommand\maketitlehookd{\vfill\null}

\title{POSIX Programming}
\author{Neil Kingdom}

\begin{document}

\begin{titlingpage}

\maketitle

\end{titlingpage}

\newpage

\tableofcontents

\newpage

\section{Abstract}

This document is intended to be a companion/extension of my C programming notes. It assumes the reader has an
intermediate level of understanding regarding the C programming language. This document explores the POSIX API
in detail. It explains some of the core principles of Unix-based systems to aid systems developers. I strongly
encourage you, the reader, to skim the POSIX reference \href{https://pubs.opengroup.org/onlinepubs/9699919799/nframe.html}{guide}
which is maintained by The Open Group. Have a look through the alphabetical listings for all of the function
definitions and become acquainted with the plethora of utilties which are available to you.

\section{Introduction}

POSIX, if you are not familiar, is a standard which most of the major Unix/BSD/Linux systems ascribe to. It
enforces standards with the aim of maximizing code portability across systems. If you use Linux or any other
Unix-like OS, you may have been under the impression that POSIX was a standard for shell scripts. While this
is technically true, POSIX is much broader in scope than simply being a standard for shell scripts. The intent
of my C programming notes was to familiarize new C programmers with the syntax and functions belonging to the
standard C library. Although this document will have some overlap with my C programming notes, for the most
part it focuses more on APIs defined in the POSIX standard such as system calls (a.k.a. syscalls) or other
OS-related utilities. Unfortunately, if you wish to become a systems developer for Windows, this guide simply
is not for you.

\subsection{A Brief History of Unix and Linux}

In the 1960's the personal computer (PC) as we know it today did not exist. Organizations such as IBM utilized
mainframes. These were monolithic pieces of machinery which used to be programmed by manually punching holes
in cards aptly called punch cards.

Operating systems did not really exist at the time. Instead, each machine would need to have its own "kernel"
code (though the concept of a kernel did not really exist at the time either) written to accommodate the
hardware which was unique to that system (e.g., the GE 645 Mainframe). As the complexity of these mainframes
grew, so did the assembly code that had to be written for them. These machines could only run one program at a
time. You would load the program into memory using a large stack of punch cards. The computer would crunch the
numbers and spit our the results to a tape or printer. Eventually, as technology advanced, multiple users
could connect to a single computer using tele-typewriters, which were essentially keyboards that could
transmit the alpha-numeric character data that was entered over long distances. We began to be able to run
more than one program at a time, and we developed assembly language which replaced the need for punch cards.

Originally, Unix was written in assembly, but after Ken Thompson and Dennis Richie had developed the C
programming language, they did a rewrite of Unix in 1973. People began to develop standards for writing
firmware. Some examples include VAX VMS, CP/M, TRON, ITRON, MuITRON, etc. Eventually, the POSIX standard was
born and would ultimately overrule its competitors.

\subsection{POSIX}

POSIX stands for Portable Operating System Interface (eXtended). This standard began in 1985 and was
officially adopted by the IEEE committee as IEEE 1003.1-1988 and ISO/IEC 9945. The 'X' in POSIX, which stands
for eXtended was introduced in POSIX.1-2001, as that version, alongside POSIX.1-2008 introduced new features
which were not present in previous versions. "But what does it mean to be 'compliant' with the POSIX standard?",
you may be asking yourself. In a nutshell, POSIX provides a list of APIs that the end user is tasked with
implementing. POSIX defines the expected inputs, outputs and general behaviour of said APIs, however, the
actual implementation may differ across systems. POSIX does not enforce things like expected level of
performance, for instance.

The POSIX standard is divided into sections based on the various components it affects. Here is a list:

\begin{itemize}

\item{%
    \textbf{POSIX.1 - Base Definitions}: Provides definitions for fundamental concepts such as constants, types,
    macros, etc.
}

\item{%
    \textbf{POSIX.1b - Real-Time Extensions}: Constitutes real-time computing features such as thread
    scheduling, pre-empting, priorities, message queues, etc.
}

\item{%
    \textbf{POSIX.1c - Thread Extensions (Pthreads)}: Establishes the baseline for POSIX threads (a.k.a.
    pthreads), thread attributes, and synchronization primitives such as mutexes and conditional variables.
}

\item{%
    \textbf{POSIX.1d - Additional File System Interfaces}: Defines the APIs used to interface with the file
    system. This includes sycalls such as open(), close(), mkdir(), link(), etc.
}

\item{%
    \textbf{POSIX.1e - Security Extensions}: Includes definitions for security-related functionality such as
    access-control lists (ACLs), authentication, authorization, etc.
}

\item{%
    \textbf{POSIX.1f - Additional Extensions}: This section contains extensions to the base POSIX standard
    that may optionally be implemented, meaning that not all systems contain these features, but those that do
    may choose to follow the POSIX standard for said features. These include things such as file system
    extensions, system performance monitoring, etc.
}

\item{%
    \textbf{POSIX.1g - Networking Interfaces}: Provides definitions for APIs pertaining to networking-related
    tasks. These include APIs such as socket(), bind(), connect(), select(), etc.
}

\item{%
    \textbf{POSIX.1-2017 - System Interfaces}: Defines most of the APIs we'll be discussing in this document.
    Provides the standard definitions for signals, file I/O, user/group management, syscalls, etc.
}

\item{%
    \textbf{POSIX.2 - Shell and Utilities}: Specifies the expected behaviour for CLI tools and shell scripts.
}

\item{%
    \textbf{POSIX.2a - Internationalization}: Provides APIs for setting the locale, system language and
    character set, time zone, etc.
}

\end{itemize}

\section{Resources and Handles}

Before we begin, I'd like to take a moment to define what is meant when we use the terms 'resource' and
'handle'. These are important terms that I'll use throughout the docment, and it will be beneficial to your
understanding if I define these terms beforehand.

A \textit{system resource} is defined to be a virtual component with limited availability. System resources
might include command line parameters, environment variables, timers, file descriptors, signal handlers, and
the program stack. Typically, a process is granted one or more resources by the kernel and then forfeits said
resources when it dies.

A \textit{handle} typically refers to either a pointer or ID which is associated with a system resource. For
instance, if I mention that a function takes in a file handle as a parameter, this most likely just means that
the function accepts some reference, pointer, or numeric ID which is associated with an open file.

\section{Processes}

I'm embarrassed to admit the amount of time it took for me to actually understand the concept of a process.
Obviously, I’m sure we’ve all heard the word 'process', but do we actually understand what that is? Put simply,
a process is an instance of a running program. In other words, it is a program in execution. In Linux, each
process is ascribed a unique process ID, or PID for short. All processes must utilize at least one thread to
execute, but if a process uses more than one thread it is considered to be a multi-threaded process. The term
'thread' is short-hand for "thread of execution". It is almost certain at the time of writing this document
that you have a processor which contains multiple cores. What this means is that the CPU contains multiple
physical dies (or copies of the CPU circuit), which are also known as cores. If a CPU has 4 cores, then
that would effectively be the equivallent of running 4 separate CPUs. Each of these cores can be considered
as a single thread of execution, and these threads are capable of executing in parallel i.e. at the same time.
The exceptions to this rule are Intel’s hyperthreading and AMD’s Simultaneous Multi-Threading (SMT) which
let each core create a virtual core known as a logical core, which effectively doubles the number of cores of
the system. Similar to processes, each thread also has its own thread ID, or TID for short. TIDs are unique
to their parent process, but not necessarily unique across the system (unlike PIDs). Processes can contain
system resources which are shared amongst the process' threads.

\subsection{The Boot Sequence}

Although it may seem irrelevant, I think it will be beneficial to understand the general steps involved in
the boot sequence for a Unix-like system. A power on sequence is initiated when the power button is pressed.
This will initiate either the BIOS/UEFI chip or Initial Program Loader (IPL) depending on the system. These
may perform some tasks such as the Power On Self Test (POST) to check the hardware before jumping to a
hard-coded address on the disk to begin executing some code. Typically, the code which lives at this hard-coded
address will be your boot loader. On Linux, the most popular boot loader is GRUB (specifically GRUB 2). GRUB
then begins what's known as the init system. The init system is the first process that runs on your system and
it has a PID of 0. The init system will fork itself into many children processes and begin several other
programs which are important for initializing the OS.

\subsection{Forking a Running Process (fork)}

\begin{clst}

pid_t fork(void);

\end{clst}

Aside from the init system which we looked at, all other processes on Unix-based systems are forked from a
parent process. This creates a sort of heirarchical process tree. The fork() system call creates a clone of
the parent process. This newly created process is known as the child process, and the calling process is known
as the parent process. The child process will be allocated stack space that is separate from the parent's
stack space, but the contents of the parent's stack will be copied to the child's stack. This does not mean
that they are exactly the same in every regard (see man pages). For instance, the child will receive a unique
PID, and regardless of whether or not the parent was multithreaded, the child process will start with a single
thread of execution. This is because it is too difficult to copy over the state of threads to a child process.
For this reason, calling fork() on multithreaded applications is discouraged. Let's analyse a bit of sample
code which utilizes the fork() syscall:

\begin{clst}

#include <stdio.h>
#include <unistd.h>
#include <assert.h>

int main(void) {
    const pid_t pid = fork();
    assert(pid != -1);

    if (0 == pid) {
        printf("Child PID: %d\n", getpid());
    } else {
        printf("Parent PID: %d\n", getpid());
    }
}

// Output (PIDs will differ on your system):
// Parent PID 15323
// Child PID 15324

\end{clst}

As soon as fork() is invoked, the process' stack along with all of its resources (file descriptors, priority,
signal mask, I/O privillege, etc.) are copied into a new stack space which is allocated for the child process.
The program's stack pointer will not change, so we resume execution in both processes immediately. In the
parent process, fork() returns the PID of the new child process, whereas in the child process, fork() returns
0. If fork() were to fail for whatever reason (e.g., not enough stack space), then it would return -1 and the
parent process would simply terminate. Assuming that fork() succeeded, the parent process will output its PID
and the child will output its PID (both retreived from a call to getpid()). If you run the program above
multiple times, you will note that sometimes the parent exits first, and sometimes the child exits first. The
order in which the print statements are executed within the example above is essentially arbitrary and comes
down to a race condition. For other programs that invoke fork(), other factors may determine the lifetime
of both the parent and child processes such as regions where the code may block awaiting a resource or user
input, or the fact that child processes do not initially inherit additional threads which are owned by the
parent.

\subsection{The Exec Family of Functions}

The exec family of functions work in tandem with fork(). Here is the list of exec functions before I proceed
with their explanations:

\begin{itemize}

\item{int execl(const char *path, const char *arg0, ..., (char *)0);}

\item{int execle(const char *path, const char *arg0, ..., (char *)0, char *const envp[]);}

\item{int execlp(const char *file, const char *arg0, ..., (char *)0);}

\item{int execv(const char *path, char *const argv[]);}

\item{int execve(const char *path, char *const argv[], char *const envp[]);}

\item{int execvp(const char *file, char *const argv[]);}

\item{int fexecve(int fd, char *const argv[], char *const envp[]);}

\end{itemize}

It may be overwhelming to see this many versions of a function, but I assure you, their differences are not
too hard to commit to memory. Before delving into the differences of each, allow me to explain the general
idea behind exec(). A useful resource in systems programming is the ability to be able to run programs from
within an already running program. fork() provides us with the ability to copy a running process, which in
turn allocates a brand new stack for us to use, copies the parent's stack, and resumes execution wherever the
stack pointer left off. However, what if we could use the stack space that was allocated for this new child
process to run a different program entirely? This is essentially what exec does. Think of it almost like
memcpy() where we copy data into an existing buffer, overwriting the previous contents, but this time we're
doing it with processes.

Let's look at the subtle differences of each version of exec.

First is execl(). This function takes a path to the binary of the application that we want to copy into our
stack created with fork() as its first argument. The second argument is the mandatory 0th argument for the
program, followed by a vararg list of additional options to give the program. The list of varargs must be NULL
terminated i.e., the last argument must be NULL. Note that in Linux, the 0th argument is always the name of
the program.

Next I'll cover execlp(). This function is identical to execl(), with the only difference being that it will
search the user's \$PATH environment variable for the binary. This means that you no longer have to provide
the absolute path to the location of the program, assuming that it exists somewhere in the user's \$PATH.

Next is execv(). This version is again, identical to execl(), with the main difference being that instead of a
list of comma separated varargs, we provide an array of arguments instead. Note that the array of arguments
must still be NULL terminated.

execvp(), as you can probably guess, combines the utility of execlp() and execv() into one. execle() is again,
identical to execl(), with the difference being that there is now an additional mandatory argument which comes
after the NULL terminator for the program arguments vararg list. This mandatory argument (env) is a another
array, similar to execv(), but for exporting environment variables at runtime. The syntax for this is how you
would expect when using the export command on Linux. For example:

\begin{clst}

char *const env[] = { "HOSTNAME=www.example.com", "PORT=8080", NULL };

\end{clst}

This will export the environment variables HOSTNAME and PORT with their respective assigned values.

execve() combines the functionality of execv() and of execle().

Finally is fexecve(). This is identical to execve(), but rather than executing a binary ELF file, it executes
whichever file is associated with the file descriptor supplied in the fd argument. The two caveats with this
are that 1. the file descriptor must be opened as O\_RDONLY or O\_PATH, and 2. the caller must have permission
to execute the file referenced by fd.

Note that child processes which do not call one of the exec functions, or that call exec but it fails, should
invoke \_exit() rather than exit(). This is so that the child process does not interfere with the parent
process' external data (files) by calling its atexit handlers, signal handlers, and/or flushing buffers.

\subsection{Virtual Fork (vfork)}

\begin{clst}

pid_t vfork(void);

\end{clst}

vfork() is an outdated function that probably should not be used anymore, but I figured we might as well
cover it. The idea behind vfork() was to optimize the rare case where you would want to call fork()
immediately followed by execve(). The idea is that execve() must copy over the entire process to the newly
reserved process space which can be a heavy operation. vfork() attempted to solve this by blocking the parent
process' calling thread and borrowing its stack space for the child process rather than allocating a new stack
and copying it from the parent. This blocking of the parent process' calling thread occurs until the child
either calls execve(), \_exit(), or terminates abnormally (e.g. via a signal). Nowadays, fork() has had many
improvements, the most notable being it's ability to do copy-on-write (COW). The way that COW works, is that
when a program requests to do a write operation, small pages of data are copied from the process to the
process' address space right before the write occurs, rather than copying over the entire process all at once.
In other words, we only copy over what's necessary when it's necessary to do so. Long story short, vfork() is
essentially deprecated and you should instead just use fork() followed by one of the exec functions.

\subsection{posix\_spawn}

\begin{clst}

int posix_spawn(
    pid_t * restrict pid,
    const char * restrict path,
    const posix_spawn_file_actions_t *file_actions,
    const posix_spawnattr_t * restrict attrp,
    char * const argv[restrict],
    char * const envp[restrict]
);

\end{clst}

One function that still is used as an alternative to exec() is posix\_spawn(). The main difference between the
exec() functions and posix\_spawn() is that we can specify file descriptors that we want the child process to
inherit from the parent.

\subsection{Zombie Processes and How to Avoid Them (wait and waitpid)}

\begin{clst}

pid_t wait(int *stat_loc);
pid_t waitpid(pid_t pid, int *stat_loc, int options);

\end{clst}

When a child process is created via fork(), it will innevitably die (exit) at some point. When a child process
dies, it sends out a SIGCHLD signal to the parent process. The parent process is able to invoke wait() or
waitpid() to determine the reason for why the child process died. In the case that the parent does not invoke
wait() or waitpid(), the child frees most of its resources, but its PID remains in the process table in order
to hold its exit status. This is known as a zombie process. Zombie processes use no CPU or resources, as they
are not running. wait() causes the calling thread to become blocked until either any child returns its exit
status, or another signal is sent telling the parent process to terminate the child. If the stat\_loc argument
is not NULL, information about the child's exit status will be stored at the location pointed to by stat\_loc.
stat\_loc will contain 0 if and only if one of the following conditions is true:

\begin{enumerate}

\item{The child process returned 0 from main()}

\item{The child process called \_exit() or exit() with a status of 0}

\item{The process was terminated because the last thread in the process was terminated}

\end{enumerate}

If the stat\_loc argument is NULL, the return status info is discarded and all zombie processes which are
descendants of the parent are cleared. waitpid() waits for any child process to return its exit status if the
argument for pid is equal to -1. In this regard, it is the same as a call to wait(). If the argument for pid
is greater than 0, it specifies the PID of a single child process for which status is requested. The stat\_loc
argument functions in the same manner as it does in wait(). The options argument can be 0 if no flags are
desired, or 1 or more of the following flags, joined through the bitwise OR operator:

\begin{itemize}

\item{%
    \textbf{WCONTINUED}: Report the status of any continued child specified by pid whose status has not been
    reported since it continued from a job control stop.
}

\item{%
    \textbf{WNOHANG}: waitpid() shall not suspend execution of the calling thread if status is not immediately
    available for one of the child processes specified by pid.
}

\item{%
    \textbf{WUNTRUNCATED}: The status of any child process specified by pid that are stopped, and whose status
    has not yet been reported since they stopped, shall also be reported to the requesting process.
}

\end{itemize}

Note that zombie processes can be avoided entirely by adding the SIGCHLD signal to the ignore list via the
signal() API:

\begin{clst}

signal(SIGCHLD, SIG_IGN);

\end{clst}

This tells the process to ignore the SIGCHLD signal completely, and thus, the child will just die instead of
becoming a zombie.

Let's take a look some sample code to get a better idea of the usage for wait()/waitpid():

\begin{clst}

#include <stdio.h>
#include <unistd.h>
#include <stdlib.h>
#include <signal.h>
#include <assert.h>
#include <sys/wait.h>

void child_died(int status) {
    printf("Child died with status: %d\n", status);
}

int main(void) {
    pid_t pid = fork();
    assert(pid != -1);

    if (pid == 0) {
        execlp("/usr/bin/sleep", "sleep", "10", NULL);
    } else {
        signal(SIGCHLD, child_died);
        waitpid(pid, NULL, 0);
    }

    return 0;
}

\end{clst}

In this example, we fork the parent process and create a new child process. Remember that fork() returns 0 for
the child process, so it invokes execlp() and replaces itself with the binary ELF for the sleep utility. The
second argument (10) is the amount of seconds to sleep for. Meanwhile, in the parent process we set up a
signal handler using signal() to run a print statement and notify us when the child dies. Immediately
afterwards, we call waitpid() which blocks the parent process until the child process dies (which happens
after 10 seconds). Once 10 seconds has expired, the child process calls exit() and the parent process can
continue execution (in this case there is nothing left to execute, so it terminates).

\subsection{Process Resources}

When a process is terminated for any reason then the process is destroyed and all OS resources that were being
used by the process are returned to the kernel. The stack is deallocated, file descriptors are closed, timers,
mutexes, semaphores, etc. are destroyed. Take the following code for example:

\begin{clst}

#include <stdlib.h>

/*
    On a computer with 16G of memory, and exactly 15G of memory
    free, how many times will this successfully run?
*/

#define ONE_GIG ((size_t)(1024 * 1024 * 1024))

int main(void) {
    void const * const p = malloc(ONE_GIG);
    return (p == NULL) ? EXIT_FAILURE : EXIT_SUCCESS;
}

\end{clst}

The answer to this question is that the program will run an infinite amount of times. Despite the program
(pressumably) allocating one gigabyte of memory using malloc each time that it is executed, once the process
is destroyed, that memory is returned to the OS for re-use, thus we never run out of memory. Additionally,
malloc() uses COW mappings, so even if this code was in an infinite while() loop, it would still run forever,
since we never actually write to the address contained in p.

\section{File I/O}

\subsection{I/O Control (ioctl)}

\subsection{File Control (fcntl)}

\subsection{Duplicating File Descriptors (dup and dup2)}

\subsection{Querying File Information (stat and fstat)}

\section{Delegating Commands (system)}

\section{POSIX Signals}

\subsection{Filtering Signals With sigset\_t}

\section{POSIX Events}

\subsection{The sigevent Struct}

\section{Sleeping}

\subsection{sleep}

\subsection{nanosleep}

\subsection{usleep}

\section{Getting the System Clock Date-Time (strftime)}

\section{Timers}

\subsection{Creating a New Timer (timer\_create)}

\subsection{Starting or Resetting a Timer (timer\_settime)}

\section{Setting the Locale}

\section{Inter-Process Communication (IPC)}

\subsection{Anonymous Pipes}

\subsection{Named Pipes or FIFOs}

\subsection{Berkley Sockets}

\section{Concurrency Via POSIX Threads}

\subsection{Scheduling Threads}

\subsection{Thread Pre-Empting}

\subsection{Creating Threads (pthread\_create)}

\subsection{Thread Attributes}

\subsection{The Thread Lifecycle}

\section{Synchronization Primitives}

\subsection{Mutexes}

\subsection{Mutex Attributes}

\subsection{Deadlock}

\subsection{Semaphores}

\subsection{Conditional Variables (condvars)}

\subsection{Reader-Writer Locks}

\subsection{Spin Locks}

\subsection{Atomic Operations}

\subsection{Others}

\section{Loading DLLs at Runtime}

\subsection{Hot-Reloading Implementation Example}

\end{document}
